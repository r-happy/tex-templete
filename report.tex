\documentclass{jsarticle}

\usepackage{rhappy}
\usepackage{arydshln}
\usepackage{jlisting}

% ---configure---
\newcommand{\TITLE}{title is here}
\newcommand{\NAME}{name is here}
\newcommand{\CLASSNUMBER}{class number is here}
\newcommand{\WEATHER}{weather is here}
\newcommand{\TEMPERATURE}{temperature is here}
\newcommand{\HUMIDITY}{humidity is here}
\title{\TITLE}
\author{}

% ---begin document---
\begin{document}

  \maketitle
  \begin{table}[b]
    \centering
    \begin{tabular}{:c:r:}
      \hdashline
      報告者 & \NAME \quad \CLASSNUMBER \\
      \hdashline
      共同実験者 & \NAME \quad \CLASSNUMBER \\
      & \NAME \quad \CLASSNUMBER \\
      & \NAME \quad \CLASSNUMBER \\
      & \NAME \quad \CLASSNUMBER \\
      & \NAME \quad \CLASSNUMBER \\
      \hdashline
      天候 & \WEATHER \\
      \hdashline
      室温 & \TEMPERATURE \\
      \hdashline
      湿度 & \HUMIDITY \\
      \hdashline
      提出日 & \today \\
      \hdashline
    \end{tabular}
  \end{table}

  \newpage

  \centerline{\Large \TITLE}
  \rightline{提出者:\quad \NAME \quad \CLASSNUMBER}
  \rightline{提出日:\quad \today}

  \section{rhappy sty package}
    \subsection{Code, CodeInline}
      It reads file from \CodeInline{public/code/.tex}. \\
      \begin{Code}[caption={here is caption}]
#include <stdio.h>
int main(void) {
    printf("this is sample code\n");
    return 0;
}
\end{Code}
    \subsection{Figure}
      This is Figure. \\
      \Figure{width=0.5\linewidth}{./public/img/kadai3-3_excel.png}{here is caption}{fig:figure_label}
      Look figure \ref{fig:figure_label}



\end{document}